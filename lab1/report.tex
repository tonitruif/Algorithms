\documentclass[12pt]{article}  
% Эта строка — комментарий, она не будет показана в выходном файле  
\usepackage{ucs} 
\usepackage[utf8x]{inputenc} % Включаем поддержку UTF8  
\usepackage[russian]{babel}  % Включаем пакет для поддержки русского языка  
\usepackage{amsmath}
\usepackage{listings}
\usepackage{color}
\usepackage{tikz}
\usepackage{pgfplots}
\usepackage{filecontents}
\usepackage{graphicx}


\definecolor{mygreen}{rgb}{0,0.6,0}
\definecolor{mygray}{rgb}{0.5,0.5,0.5}
\definecolor{mymauve}{rgb}{0.58,0,0.82}
\title{Отчет по лабораторной работе 1 \\ 
	По предмету “Анализ алгоритмов” \\
	По теме “Расстояния Левенштейна и Дамерау-Левенштейна”
}  
\date{2018}  
\author{Фирсова Дарья ИУ7-56}

\begin{document}  
  \maketitle  
  \newpage
\section*{Введение}
В лабораторной работе изучаются расстояние Левенштейна и расстояние Демерау-Левенштейна. Требуется применить метод динамического программирования,  изучить работу алгоритма и получить практические навыки реализации алгоритмов. Задачи для лабораторной работы: 


\end{document}